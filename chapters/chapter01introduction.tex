%!TEX root = ../main.tex
\chapter{Introduction}
\label{chap:intro}

\section{Problem Definition}
Parents can often find it difficult to motivate their children to perform chores in a timely manner. 
Many parents have attempted to implement a simple reward system to incentivize their children, but often fail to keep up with the rewards or maintain the tracking needed to make the positive reinforcement truly effective.
I believe this is a problem that all parents face, and one we have all faced as children ourselves.  

\section{Proposed Artefact}
I want to create a mobile application that will allow parents to assign tasks and rewards to their children in the form of `quests' in a role-playing game, effectively gamifying chores. 
The quests can take the form of ``Tidy your room'' or ``Complete your homework'' and offer rewards of experience points (XP) and gold. The application will also allow the child to create an in-game character that they can customise and level up using aforementioned gold and XP.
I believe the app will provide children with incentive and positive reinforcement to acheive more.

%motivations for it

Inspiration from this project came from two main sources: Firstly, the online game Habitica, which sought to help people better themselves by adding their own tasks and to-do lists as quests and offering self-defined rewards on completion. 
Habitica, however, focuses largely on adults and requires people to manage the software themselves, whereas this application will provide two separate interfaces for the adult and child to help parents motivate their children; allowing the parents to tailor the software to the specific child.

Secondly, an experiment at Indiana University, where a professor chose to give out XP rewards and levels in his Games Development class instead of letter based grades \cite{sheldon2011multiplayer}.  
The professor found increased enthusiasm and participation amongst students in the class \cite{sheldonclasspostmortem}.

\section{Clients}
The key clients for this app will be the parent(s) or guardians inputting quests into the game and marking a quest as complete once they have inspected the work. 
The child will also be able to view quests, notify the parents that a quest requirement is ready for inspection, and choose skills and equipment for their character.
Parents will also be able to monitor their children's activities and progress, whilst approving or rejecting various activities that the child may encounter whilst using the application. 

\section{Aims and Objectives}
The aim of this project is to evaluate the usefulness of game-based rewards for children performing chores. 
To achieve this, I have identified the following objectives for the software artefact:

\begin{itemize}
	\item Create a well designed mobile application, capable of tracking progress and offering rewards for chores and adhering to the Android Developer specifications
	\item Create a secure and robust public API to allow applications to connect to the the artefact's service
	\item Utilize appropriate technologies to securely create the two deliverables to a high quality
	\item Utilize test-driven development to provide automated testing, ensuring test coverage of over 80\% for the API
\end{itemize}

\section{Risks}
In this project, I will be working with a variety of technologies that I have not previously used or am not very experienced with, such as Flask, Android and REST. 
This presents a likely risk that the learning curves of these tools may be steeper than expected.
If this occurs, it is likely that initial time-frame estimates for the development will be inaccurate, as features involving these technologies may incur delays.
The scale and importance of the above delays could, in themselves, lead to the project not entirely fulfilling the above objectives. I believe this makes the above risk high severity.
In order to mitigate this, I have researched a variety of different learning materials that I will be able to access throughout the project to help me better understand these tools.
Furthermore, I have sourced potential help from my placement company, where colleagues have agreed to provide guidance in areas they are experienced in.

As this is a task management app, the artefact risks becoming more of a hindrance then a help.
There is a possibility that the app becomes too intrusive to parents giving their children a task, as what would normally be a simple request could be delayed by the additional process of entering and tracking the task.
If this was to occur, it would result in the app being less desirable to use.
However, I believe that if I put a sufficient focus on usability within the app and attempt to design my use cases to have as little resistance as possible, I can avoid this risk.
I can also lower this risk by offering Parent users their own version of the app, which would allow them to circumvent any security features designed to stop a Child from giving themselves rewards. 

Another risk that may arise is the potential for hardware failure. 
Should it occur, it could wipe out the code-base and the project report, effectively destroying the project. 
Whilst it is unlikely, steps must still be taken to mitigate this, and so I will store both the code and the report in multiple local and remote locations.
The report, app code and server code will all be stored on both my PC and my Laptop as a local backup, and all three will also be stored in a remote GitHub repository (Which will remain private until the deadline to avoid plagiarism).
This will ensure that should the worst occur, I will not lose any work barring the work that is yet to be uploaded.
Therefore if I make regular commits to GitHub, the deliverables of this project shall remain safe. 

\section{Terminology}

The following terms will be used throughout:

\begin{itemize}
	\item Parent - The primary human user. The Parent assigns, manages, and approves their child's quests. In actual usage, this will usually be a real-life parent or guardian of the secondary user. 
	\item Child - The secondary human user. The Child marks quests as complete and creates an character. In actual usage, this will usually be a dependent of the primary user. 
	\item Quest - A task given to a child.
	\item Quest Reward - Experience Points (XP) or gold. These can be used to level up a character or purchase real-life rewards.
	\item RPG - Role playing game, a common format of video games where a focus is put on levelling up (strengthening) you character.
\end{itemize}