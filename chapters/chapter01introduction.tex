%!TEX root = ../main.tex
% What is the project about? 
% What problem are you tackling? 
% What is your research question? 
% Why do these problems need solutions? Why are they important?
% What is the background to the problem? Who is the client? What do they want?
% What existing methods have been tried? How has I.T. been applied so far? 
% What constraints do you have? (Time, PCs, money, users, software etc) 
% What is the scope of what you have set yourself to do. What is not included?
% What broad approach was taken? (Summarise your broad approach the project) 
% Risk management summary

\chapter{Introduction}
\label{chap:intro}

\section{Problem Definition}
Parents often find it difficult to motivate children to perform chores in a timely manner.

I want to create a mobile application that will allow parents to assign tasks and rewards to their children in the form of ``quests'' in a role-playing game, effectively gamifying chores. 
The quests can take the form of “Tidy your room” or “Complete your homework” and offer rewards of experience points and gold to level up the child's in-game avatar. 
I believe the app will provide children with incentive and positive reinforcement to acheive more.

Inspiration from this project came from two main sources: Habitica (Previously: HabitRPG), which sought to help people better themselves by adding their own tasks and to-do lists as quests and completing them to earn self-defined rewards. 
However, this game focuses largely on the self-improvement of adults and required people to self manage the software, whereas KidQuest will provide the separation of adult and child to help parents motivate their children.

Secondly, an experiment at Indiana University, where a professor chose to give out XP rewards and levels instead of letter based grades \cite{sheldon2011multiplayer} 

The key clients for this app will be the parent(s) inputting quests into the game and marking a quest as complete once they have inspected the work. 
The child will also be able to view quests, notify the parents that a quest requirement is ready for inspection and choose skills and equipment for their character. 

I chose to complete this project by developing an Android application to provide the functionality and a Python webserver to perform required data analysis.

\section{Objectives}

\section{Proposed Artefact}

\section{Risks}

%Add something about RPG elements