%!TEX root = ../main.tex
% What is the project about? 
% What problem are you tackling? 
% What is your research question? 
% Why do these problems need solutions? Why are they important?
% What is the background to the problem? Who is the client? What do they want?
% What existing methods have been tried? How has I.T. been applied so far? 
% What constraints do you have? (Time, PCs, money, users, software etc) 
% What is the scope of what you have set yourself to do. What is not included?
% What broad approach was taken? (Summarise your broad approach the project) 
% Risk management summary

% 1000-2000 words

\chapter{Introduction}
\label{chap:intro}

\section{Problem Definition}
Parents often find it difficult to motivate children to perform chores in a timely manner. 
Many parents have attempted to implement a simple reward system to incentivize their children, but often fail to keep up with the rewards or maintain the tracking needed to make the positive reinforcement truly effective.
I believe this is a problem that all parents face, and one we have all faced as children ourselves.  

\section{Proposed Artefact}
I want to create a mobile application that will allow parents to assign tasks and rewards to their children in the form of `quests' in a role-playing game, effectively gamifying chores. 
The quests can take the form of ``Tidy your room'' or ``Complete your homework'' and offer rewards of experience points and gold to level up the child's in-game avatar. 
I believe the app will provide children with incentive and positive reinforcement to acheive more. test

Inspiration from this project came from two main sources: the online game Habitica, which sought to help people better themselves by adding their own tasks and to-do lists as quests and offering self-defined rewards on completion. 
However, this game focuses largely on adults and required people to manage the software themselves, whereas KidQuest will provide the separation of adult and child to help parents motivate their children.

Secondly, an experiment at Indiana University, where a professor chose to give out XP rewards and levels in his Games Development class instead of letter based grades \cite{sheldon2011multiplayer}  and found increased enthusiasm and participation amongst students in the class \cite{sheldonclasspostmortem}.

\section{Clients}
The key clients for this app will be the parent(s) inputting quests into the game and marking a quest as complete once they have inspected the work. 
The child will also be able to view quests, notify the parents that a quest requirement is ready for inspection and choose skills and equipment for their character. 
Parents will also be able to monitor their children's activities and progress, whilst approving or rejecting various interactions that the child may have on the app.

\section{Aims and Objectives}
The aim of this project will evaluate the usefulness of game-based rewards for children performing chores. 
To achieve this, I have identifeid the following objectives for KidQuest:

\begin{itemize}
	\item Create a well designed mobile application, capable of tracking and rewarding children for chores and adhering to the Android Developer specifications.
	\item Create a secure and robust public API to allow applications to connect to the KidQuest service.
	\item Utilize appropriate technologies to securely create the two deliverables to a high quality
	\item Utilize automated testing to ensure test coverage of over 80\% for the API.
\end{itemize}

\section{Risks}
As this is a solo development project, there is a significant risk that there will be flaws in the initial design. 
This is due to the fact - coupled with the fact that this is new technology to me - that there is a high probability that I will misunderstand what is needed for certain designs or misunderstand what the correct tool for the job is.
Because of this I would argue that during this project, it is almost certain that the initial spec for this project will change and redesigns will need to be made. 
In order to mitigate this, I have taken steps to make individual features and sections of the code of the project as modular as possible, hoping to be able to swap out sections of the code should certain changes arise.
I will also code many aspects of the app with the idea that it may move to a server back-end, rather than having data stored in the app. 
 
The risk of changes in turn raises a key risk to the project which is that it may not be ready for the deadline. 
I consider this to be a high severity risk to the project, as the hand-in date for the project is firm and unchangeable. 

As I am the only developer on the project, there is a high risk of the code quality being lower than what can be expected from a team.
I am undertaking a self code-review policy, where I reserve some time to inspect the code of each feature branch before it is merged into the master branch.
I believe that if I am sufficiently strict with the reviews, they will help me ensure that code quality is kept up to an acceptable standard.
Arrangements have also been made with a fellow course member to code review each others code as our projects are sufficiently different enough to avoid plagiarism and we both have strong interests in keeping our code quality to a high standard.
When utilized correctly, code review can find approximately 30-70\% of logic errors within a program \citep{myers2011art}.

Another risk that may arise is the potential for hardware failure. 
Should it occur, it could wipe out the codebase and the project report, effectively destroying the project. 
Whilst it is unlikely, steps must still be taken to mitigate this, and so I will store both the code and the report in multiple local and remote locations.
The report, app code and server code will all be stored on both my PC and my Laptop as a local backup, and all three will also be stored in a remote GitHub repository (Which will remain private until the deadline to avoid plagiarism).
This will ensure that should the worst occur, I will not lose any work barring the work that is yet to be uploaded.
Therefore if I make regular commits to GitHub, the deliverables of this project shall remain safe. 