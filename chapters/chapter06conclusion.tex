%!TEX root = ../main.tex
%500-1000 Words
\chapter{Conclusion}
\section{Features}
Allowing Parent users to log into the app separately from a Child user was an overall success and helped to limit the intrusiveness outlined in my risk analysis.
The use of a server API back-end also allowed the two users' apps to update from each other very smoothly, and in a manner that is seamless to the user.

The addition of the reward shop is very useful to the system, as it provided a real-life reward element to the app.
I believe this will be the key driving force in the adoption of the app, as it will provide the backbone for the motivational aspects of the RPG style rewards.

\section{Personal Appraisal}
I believe the system to be a very good proof of concept of the application, providing all of the important functionality detailed in the requirements chapter.
The project has also allowed me to gain a great deal of knowledge into the development of an Internet connected mobile application as well as API and web service development, which will be invaluable going forward.

\section{Evaluation}
Upon evaluating the project, I feel that my initial decision to exclude outside participants for ethics reason was a mistake. 
I believe that my evaluation of the software artefact suffered from a lack of structured external analysis and would have benefited from feedback such as beta testing or usability testing. 

I also feel that choosing to make the system as a native Android app may not have been the best choice for this project.
Whilst using a API back-end has improved the systems portability, a move to a web application would theoretically provide complete portability.
Using a responsive design website, I would have been able to create an application that would be accessible from any device on any OS.
However, having a native app also comes with several of its own benefits such as efficiency and scalability, as using a website would have put much more strain on the server application.

\section{Future Work}
I fully intend to continue work on the application, though I believe it would require more work to be considered complete, as many features in the `Should Have' and `Could Have' lists did not make it into the final version of the artefact.
There are a number of social media based features that would be beneficial to the software, such as the `Friend Battles' feature that would have introduced an element of competition into the motivational features.
It is also my opinion that a complete version of the application would benefit greatly from a full aesthetic overhaul, including graphics, animations, and sound effects, to make it more appealing to younger children.
Once the application is considered feature complete, the application could be made available for public beta testing.

Despite many things that I may have chosen to differently in this project, I feel that the the application was an overall success and achieved the aims and objectives set out at the beginning of the project.