%!TEX root = ../main.tex
\chapter{Requirements and Analysis}
\label{chap:methodology}

There are two main users for KidQuest, the child and the parent. 
It is important that the requirements of both users are analysed separately to insure that the application is suitable for both.

\section{The Child}
The child will open the application and be met with a character screen, which will display the relevant information about their character, including the XP/Gold they currently have, the XP needed until the next level and the recent rewards they have earned.
The child can then check their quests and mark any that they have finished as complete.
The quest will then move into the pending quest log, ready for a parent to confirm.
Once the parent has confirmed the quest, the child will be met with a reward of experience points and gold based on the difficulty of the quest.

The child can visit the rewards section to spend their gold on in-game items which will increase the power level of their character. 
Alternatively, they can spend their gold on rewards that have been set by the parents. 
i.e. 400 gold will be able to purchase a visit to the cinema.

The application controls must be kept simple and uniform so that they are able to be learned and used by children.
Children will not be able to access any parent functions as they will require the parent's pin code. 

\section{The Parent}
The parent can then either use the child's application (with a pin code) or use their own parent edition of the application to approve the completion.

The parent can create a quest with 5 different difficulty levels ranging from very hard to very easy, which the game will automatically generate a reward based on the difficulty. 
This is to minimize the level of interaction required from the parent.

Don't do punishments. only successes. not my place.