%!TEX root = ../main.tex
\chapter{Implementation}

As the project began it was clear that my initial database/class diagrams were drawn with incorrect assumptions about the project. 
Previously, I have mostly worked on systems that handle many users logging into a web application and designed databases as such. 
However, I did not account for the fact that with an android application, the database only needs to handle one user, which allowed me to use a simpler - less relationship heavy - database design.

\section{Server Rewrite}
When it came to be time to implement the parent app, it was realized that the initial designs for the server were not suitable for KidQuest.
I had initially tried to implement all workflows into the app and store the quests on the client end. 
The plan was then to send the quests to the parent phone using Google Cloud Messaging (GCM) and have them send back which quests were completed.
However, GCM was not suitable for moving the quests reliably and I found that it was too likely that information would be lost or become out of sync between parent and child devices.

Therefore, I decided to push most of the data and functionality to the server-side and have the app interact with the server to perform actions.
I still chose to use GCM for communicating from the server to a client in order to push notifications when tasks are confirmed or completed.

However, making this move did have two main consequences. 
Firstly, users would now require an internet connection to use the app.
Data implications.
Security requirements.
Queueing messages to server for low connectivity.

Secondly, all users were now required to create an account with the server in order to use the app.
I believe this is 

Separating child and parent users was a mistake.