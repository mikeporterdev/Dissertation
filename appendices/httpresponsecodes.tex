%!TEX root = ../main.tex
\chapter{HTTP Response Codes}

Upon receiving and processing a request, the server will return a response code to inform the user if the request was accepted or rejected.
The codes are designed by \cite{rfc7231} to give the sending machine context on what happened to the request.
For example, a user's phone can send a quest to the server and check for the response code.
If the code is 201, the phone knows it can show a success message and tell the user the quest was saved.
A brief explanation of the response codes used in the server are listed below.

\bigskip

\begin{tabularx}{\linewidth}{|p{3cm}|X|}\hline
\multicolumn{1}{|c|}{\textbf{Response Code}} & \multicolumn{1}{c|}{\textbf{Description}}                                                              \\ \hline
200 Success                                  & Request has been accepted and is successful.                                                           \\ \hline
201 Created                                  & Object has successfully been created in the server.                                                    \\ \hline
400 Bad Request                              & Request contains errors or is attempting to access an object that does not exist.                      \\ \hline
401 Not Authorized                           & Incorrect email or password, or not authorized to perform actions on that account.                     \\ \hline
404 Not Found                                & URL does not exist on server.                                                                          \\ \hline
405 Method Not Allowed                       & Request method (GET/POST/PUT etc.) is not supported on this endpoint.                                  \\ \hline
409 Conflict                                 & Request could not be processed due to a conflict, e.g. Creating a user with email that already exists. \\ \hline
500 Internal Server Error                    & An error has occured within the server that means the request could not be processed.                  \\ \hline
\end{tabularx} 